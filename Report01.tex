\documentclass[a4paper,12pt]{article}
\usepackage[framed,numbered,autolinebreaks,useliterate]{mcode}
\usepackage{CJKutf8}
\setlength{\headheight}{15pt} 

\usepackage{textcomp}
\usepackage{amsmath}
\usepackage{amssymb}
\usepackage{listings}
\usepackage{float}
\usepackage{xcolor}
\usepackage{color}
\usepackage{fancyhdr}
\usepackage{lastpage}
\usepackage{times}
\usepackage{mathptmx}
\usepackage{geometry}
\usepackage{booktabs}
\usepackage{graphicx}
\geometry{left=3.17cm,right=3.17cm,top=2.54cm,bottom=2.54cm}
\usepackage{indentfirst}
\setlength{\parindent}{2em}
\rhead{Page \thepage{} of \pageref{LastPage}}
%\pagestyle{fancy}



\begin{document}
\begin{CJK*}{UTF8}{gbsn}



\section{实验课题}
在研究飞机的自动着陆系统时,技术人员需要分析飞机的降落曲线.根据经验,一架水平飞行的飞机,其降落曲线是一个三次多项式.设飞机飞行的高度为$h$,飞机着陆点为原点,且在整个着陆过程中,飞机始终保持水平飞行姿势,且水平速度保持为常数.出于安全考虑,飞机垂直加速度的最大绝对值不得超过$g/10$,此处$g$是重力加速度,进行以下实验:\par 
\begin{enumerate}
\item 若飞机从距降落点水平距离$l$处开始降落,确定飞机的降落曲线.假设飞机降落参数为:飞机的水平速度$u=540$km,$h=10$km,$l=80$km,绘出飞机降落曲线图形.
\item 求出飞机能够安全降落时水平距离所能允许的最小值.
\item 假设飞机跑道长度为$3.6$km,已知飞机着陆后匀减速在跑道上滑行,直到完全停下,应用上面给出的飞行参数模拟飞机降落的全过程.
\end{enumerate}



\section{问题一}


\subsection{参数和变量列表}
\begin{center}
\textbf{表一}\quad 实验一参数变量表\\\vspace{2pt}
\begin{tabular}{cclc}
\toprule[1.5pt]
& 符号 & \qquad\qquad\quad含义 &\\
\midrule[1.5pt]
& $l$ & 飞机距降落点的水平距离 &\\
& $h$ & 飞机的初始飞行高度 &\\
& $x$ & 飞机位置的横坐标 &\\
& $f(x),y$ & 飞机纵坐标关于横坐标$x$的函数 &\\
& $g$ & 重力加速度 &\\
\bottomrule[1.5pt]
\end{tabular}
\end{center}


\subsection{数学推导}
设飞机的降落曲线为$y=f(x)=ax^3+bx^2+cx+d$,以飞机着陆点为原点建立平面坐标系,则有
\begin{equation}
f(-l)=h,f(0)=0. \label{eq1}
\end{equation}
又因为飞机在降落过程中始终保持水平飞行,故有
\begin{equation}
f'(-l)=0,f'(0)=0. \label{eq2}
\end{equation}
由\eqref{eq1}和\eqref{eq2}得$a=2h/l^3$,$b=3h/l^2$,$c=0$,$d=0$.所以飞机的降落曲线为
\begin{equation}
f(x)=\frac{2h}{l^3}x^3+\frac{3h}{l^2}x^2. \label{eq3}
\end{equation}
代入$u$,$h$,$l$的具体数值,我们得到$f(x)$的数值表达式
\begin{equation*}
f(x)=\frac{1}{25600}x^3+\frac{3}{640}x^2.
\end{equation*}


\subsection{算法实现}
\textsc{Matlab}提供了solve函数可以帮助我们完成方程的求解.
\vspace{-11pt}
\begin{lstlisting}
R = solve(A*(-l)^3+B*(-l)^2+C*(-l)+D==h,D==0,...
    3*A*(-l)^2+2*B*(-l)+C==0,C==0,A,B,C,D);
r1 = double(subs(R.A,{l,h},{80,10}));
r2 = double(subs(R.B,{l,h},{80,10}));
r3 = double(subs(R.C,{l,h},{80,10}));
r4 = double(subs(R.D,{l,h},{80,10}));
F = [r1 r2 r3 r4]
\end{lstlisting}
\vspace{7pt}
\par 其中$R$是一个表示计算结果的向量,其中每一个元素都是关于$l$和$h$的.此时我们把$l=80$km和$h=10$km代入$R$的四个分量就可以得到$F$.F是一个表示$f(x)$的多项式系数的矩阵.
\section{问题二}


\subsection{参数和变量列表}
\begin{center}
\textbf{表二}\quad 实验二参数变量表\\\vspace{2pt}
\begin{tabular}{cclc}
\toprule[1.5pt]
& 符号 & \qquad\qquad\qquad\qquad含义 &\\
\midrule[1.5pt]
& $l$ & 飞机距降落点的水平距离 &\\
& $u$ & 飞机在水平方向上的速度 &\\
& $v$ & 飞机在垂直方向上的速度 &\\
& $h$ & 飞机的初始飞行高度 &\\
& $x$ & 飞机位置的横坐标 &\\
& $f(x),y$ & 飞机纵坐标关于横坐标$x$的函数 &\\
& $g$ & 重力加速度 &\\
& $a_y$ & 飞机在点$(x,y)$处的垂直加速度 &\\
& $l_{min}$ & 飞机能够安全降落所允许的最小水平距离 &\\
\bottomrule[1.5pt]
\end{tabular}
\end{center}


\subsection{数学推导}
由题意我们知道飞机在平方向上的速度$u=\displaystyle{\frac{{\rm d}x}{{\rm d}t}}$为常数.根据复合函数求导法则对\eqref{eq3}关于$t$求导,得飞机在垂直方向上的速度
\begin{equation}
v=\frac{{\rm d}y}{{\rm d}t}=\frac{{\rm d}y}{{\rm d}x}\frac{{\rm d}x}{{\rm d}t}=u\frac{{\rm d}y}{{\rm d}x}. \label{eq4}
\end{equation}
对\eqref{eq4}关于$t$再求导,我们可以得到飞机在点$(x,y)$处的垂直加速度
\begin{equation}
a_y=\frac{{\rm d}v}{{\rm d}t}=\frac{u{\rm d}(\displaystyle{\frac{{\rm d}y}{{\rm d}x})}}{{\rm d}t}=u^2\frac{{\rm d^2}y}{{\rm d}x^2}=\frac{6hu^2}{l^3}(l+2x). \label{eq5}
\end{equation}
又飞机在降落过程中有$x\leqslant0$恒成立,由\eqref{eq5}易知当$x=0$时$a_y$取到最大值$\displaystyle{\frac{6hu^2}{l^2}}$.又由题意,为使乘客感到舒适,降落全过程中,加速度$a_y$不能超过$g/10$.所以有
\begin{equation}
{\frac{6hu^2}{l^2}\leqslant\frac{g}{10}}. \label{eq6}
\end{equation}
由\eqref{eq6}我们可以得到水平距离$l$可以取到的最小值
\begin{equation}
l_{min}=2\sqrt{\frac{15h}{g}}u. \label{eq7}
\end{equation}
把$u$,$h$,$g$的值带入\eqref{eq7},我们得到飞机能够安全降落时水平距离的最小值
\begin{equation*}
l_{min}=37.1154{\rm km}.
\end{equation*}


\subsection{算法实现}
由于\textsc{Matlab}求解含待定系数和约束条件的不等式的能力不够完美,这里采用的是先把不等式转化成等式,再使用solve函数求解的方法.
\vspace{-11pt}
\begin{lstlisting}
l_min = solve(subs(6*h*u^2/l^2+12*h*u^2*x/l^3==g/10,x,0),l);
l_min = vpa(subs(l_min(1),{h,u,g},{10,540/3600,0.0098}))
\end{lstlisting}
\vspace{7pt}
\par 按照上面的程序求得的$l_{min}$有两个,一正一负.由题意我们保留正根,舍去负根.

\section{问题三}


\subsection{参数和变量列表}
\begin{center}
\textbf{表三}\quad 实验三参数变量表\\\vspace{2pt}
\begin{tabular}{cclc}
\toprule[1.5pt]
& 符号 & \qquad\qquad\qquad\qquad含义 &\\
\midrule[1.5pt]
& $l$ & 飞机距降落点的水平距离 &\\
& $u$ & 飞机在水平方向上的速度 &\\
& $x$ & 飞机位置的横坐标 &\\
& $f(x),y$ & 飞机纵坐标关于横坐标$x$的函数 &\\
& $t$ & 飞机在跑道上的滑行时间 &\\
& $x(t)$ & 飞机横坐标坐标关于滑行时间$t$的函数 &\\
& $T_1$ & 飞机的着陆时间(不包括在跑道上的滑行时间) &\\
& $T_2$ & 飞机的滑行总时间 &\\
& $d$ & 飞机跑道的长度 &\\
& $a$ & 飞机匀减速滑行的加速度 &\\
\bottomrule[1.5pt]
\end{tabular}
\end{center}


\subsection{数学推导}
飞机的整个降落过程包括两个阶段:降落阶段和滑行阶段.降落阶段的曲线已由前面的实验求出,并且易求得降落时间$T_1=l/u$.现在求滑行阶段的曲线.由于在滑行阶段飞机没有竖直方向上的位移,我们只需要考虑飞机的水平位移$x$和时间$t$的关系,记为$x(t)$.根据物理学匀减速运行公式,我们设加速度为飞机加速度为$a$,所以有
\begin{equation}
x(t)=ut-\frac{1}{2}at^2. \label{eq8}
\end{equation}
设飞机滑行时间为$T_2$,由题设条件不难得到
\begin{equation}
x(T_2)=d,x'(T_2)=0. \label{eq9}
\end{equation}
解\eqref{eq8}\eqref{eq9}得飞机的滑行方程
\begin{equation}
x(t)=ut-\frac{u^2}{4d}t^2. \label{eq10}
\end{equation}
把$u$,$d$的数值代入\eqref{eq10},所以我们有
\begin{equation*}
x(t)=540t-20250t^2.
\end{equation*}
\subsection{算法实现}
首先使用solve解出飞机滑行时间和加速度.进而我们可以求出飞机的纵坐标和横坐标关于$t$的函数关系,为作图做准备.
\vspace{-11pt}
\begin{lstlisting}
Q = solve(u*T-a*T^2/2==d,u-a*T==0,a,T);
q1 = double(subs(Q.a,{d,u},{3.6,540}));
q2 = double(subs(Q.T,{d,u},{3.6,540}));

F_x_t = [540 -80];
F_y_t = sym2poly(subs(3*h*x^2/l^2+2*h*x^3/l^3,...
    {h,x,l},{10,540*t-80,80}));

G_x_t = [-540^2/(4*3.6) 540 0];
G_y_t = 0;
\end{lstlisting}
\vspace{7pt}
\par 分两段画出$f(x)$的图像,第一段表示飞机的降落曲线,第二段表示飞机的滑行曲线.
\vspace{-11pt}
\begin{lstlisting}
xx = linspace(-80,0,5000);
yy = polyval(F,xx);
g1 = plot(xx,yy);
hold on;

xx = linspace(0,3.6,300);
yy = polyval(G,xx);
g2 = plot(xx,yy);
hold on;
\end{lstlisting}
\vspace{7pt}
\par 使用如下的While循环可以画出动点模拟飞机的降落.
\vspace{-11pt}
\begin{lstlisting}
while 1
    if ~ishandle(h),return,end
    if (t<T1)
        set(p1,'xdata',polyval(F_x_t,t),'ydata',polyval(F_y_t,t));
    elseif ( t>=T1 && t<=T2 )
        set(p1,'xdata',polyval(G_x_t,(t-T1)),'ydata',...
        polyval(G_y_t,(t-T1)));
    else
        t = 0;
    end
    t = t + dt;
    drawnow
end
\end{lstlisting}
\vspace{7pt}
\par 最后我们可以得到飞机降落过程的模拟图.
\begin{center}
\includegraphics[width = 13.5cm]{plane.jpg}\\
\textbf{图一} 飞机降落曲线动态模拟截图\\
\end{center}
\par 在\textbf{图一}中绿色的曲线表示飞机在空中的运动轨线,红色的曲线表示飞机在跑道上的运动轨线,红点表示飞机的位置.
\section{实验总结}
本实验主题是飞机降落曲线的求解和降落过程的模拟,并涉及多项式计算和微分方程求解,并考察了较多\textsc{Matlab}操作,如解方程和作曲线图.\par 从数值计算和算法设计角度来看,本实验难度不高,但涉及了不少\textsc{Matlab}基本操作,并不是毫无工作量的.\par 此外,在这次实验中,\textsc{Matlab}也暴露出它较之于\textsc{Mathematica}和\textsc{Maple}的一些不足,比如在求解待定系数不等式方面.

\section{附录}
\noindent\textbf{Lab01.m} 源代码
\vspace{-10pt}
\lstset{basicstyle=\ttfamily\footnotesize,escapechar=`}
\begin{lstlisting}
syms x;
syms A;
syms B;
syms C;
syms D;
syms l;
syms h;
syms u;
syms g;
syms T;
syms a;
syms d;
syms t;
format long;

R = solve(A*(-l)^3+B*(-l)^2+C*(-l)+D==h,D==0,...
    3*A*(-l)^2+2*B*(-l)+C==0,C==0,A,B,C,D);
r1 = double(subs(R.A,{l,h},{80,10}));
r2 = double(subs(R.B,{l,h},{80,10}));
r3 = double(subs(R.C,{l,h},{80,10}));
r4 = double(subs(R.D,{l,h},{80,10}));
F = [r1 r2 r3 r4]

l_min = solve(subs(6*h*u^2/l^2+12*h*u^2*x/l^3==g/10,x,0),l);
l_min = vpa(subs(l_min(1),{h,u,g},{10,540/3600,0.0098}))

Q = solve(u*T-a*T^2/2==d,u-a*T==0,a,T);
q1 = double(subs(Q.a,{d,u},{3.6,540}));
q2 = double(subs(Q.T,{d,u},{3.6,540}));
G = 0;
T1 = double(80/540);
T2 = T1+q2;

F_x_t = [540 -80];
F_y_t = sym2poly(subs(3*h*x^2/l^2+2*h*x^3/l^3,...
    {h,x,l},{10,540*t-80,80}));

G_x_t = [-540^2/(4*3.6) 540 0];
G_y_t = 0;

h=figure('numbertitle','off','name',' Lab01');

xx = linspace(-80,0,5000);
yy = polyval(F,xx);
g1 = plot(xx,yy);
hold on;

xx = linspace(0,3.6,300);
yy = polyval(G,xx);
g2 = plot(xx,yy);
hold on;

set(g1,'Linestyle','-','color','g','Linewidth',1.5);
set(g2,'Linestyle','-','color','m','Linewidth',1.5);
xlabel('x');
ylabel('y');
axis([-90,10,0,10.05]); 

x0 = -80;
y0 = 10;
p1 = plot(x0,y0,'color','r','marker','.','markersize',15); 
t = 0;
dt = double(80/540000);
while 1
    if ~ishandle(h),return,end
    if (t<T1)
        set(p1,'xdata',polyval(F_x_t,t),'ydata',polyval(F_y_t,t));
    elseif ( t>=T1 && t<=T2 )
        set(p1,'xdata',polyval(G_x_t,(t-T1)),'ydata',...
        polyval(G_y_t,(t-T1)));
    else
        t = 0;
    end
    t = t + dt;
    drawnow
end
\end{lstlisting}




\end{CJK*}
\end{document}
